\documentclass[a4paper,oneside,12pt]{article}
\usepackage[utf8]{inputenc}
\usepackage{graphicx}
\usepackage{t1enc}
\usepackage[magyar]{babel}
\usepackage{hyperref}
\usepackage{pdflscape}
\usepackage{times}
\usepackage{listings}
\lstset{language=Java}
\usepackage{multirow}
\selectlanguage{magyar}
\frenchspacing
\linespread{1.3}
\setlength\lefthyphenmin{2} % legalább két karakter a kötőjel előtt
\setlength\righthyphenmin{2}% legalább két karakter a kötőjel után
\setlength\hyphenpenalty{0} % nem baj, ha elválasztunk a sor végén
\usepackage{fancyhdr}
\setlength{\headheight}{15.2pt}
\pagestyle{fancy}
\lhead{Fejlesztői dokumentáció}  
\rhead{Likker Ádám -- OXWQXX}


\hypersetup{
    colorlinks,
    citecolor=black,
    filecolor=black,
    linkcolor=black,
    urlcolor=blue
}

\begin{document}
\begin{titlepage}
	\pagestyle{empty}
	\begin{center}
		{\vfill \bf {\LARGE Küldemény nyilvántartó rendszer \\Fejlesztői dokumentáció\\[60pt]} \Large Szoftverfejlesztés J2EE platformon\\[90pt] Likker Ádám -- OXWQXX}
	\end{center}
	\newpage	
	\tableofcontents
\end{titlepage}
\section{A feladat leírása}
Kézzel kitöltött jelentések (Delivery Note-ok):
\begin{itemize}
	\item Nem egyszerűen hozzáférhető Trouble Report-ok
	\item Rengeteg régi, fel nem dolgozott Delivery Note
\end{itemize}
%
\begin{lstlisting}
private String pwd;
private String tel;
	private Boolean dispatcher;
\end{lstlisting}
Elérkeztünk egyben a \textbf{listákhoz}, valamint a \emph{for} ciklushoz. Listák esetén többféle elem is megadható vegyesen a tömb elemeiként, illetve látható, hogy dinamikusan kezeli a Python még az egyszerű adattípusokat is -- \emph{append() segítségével új elem adható a tömbhöz}.
\begin{lstlisting}
>>> for i in range(0,5):
		array.append(i)
>>> print array
		[0, 1, 2, 3, 4]
>>> array[2]='ez egy tomb'
>>> print array
		[0, 1, 'ez egy tomb', 3, 4]
\end{lstlisting}
A \emph{while} ciklusban pedig megadjuk a logikai kifejezést a kulcsszó után. Sokkal érdekesebb a \emph{for} ciklus viselkedése: alap működése során egy objektum elemein halad végig. Fent a \emph{range} változóval adom meg, milyen tartományban fusson. Látható a Python szintaxisa is: egy blokk határait nem kapcsos zárójelek határolják, hanem négy szóköz vagy egy tabulátor. Valamint egy sor egy utasítás. Így rákényszerítik a programozót átlátható, szépen formázott kód írására.

\textbf{Tuple/rekord} butított tömbként viselkedik: tartalma nem módosítható (csak létrehozás, törlés).
\begin{lstlisting}
>>> tup1 = (50,)
>>> tup2 = tup1 + (69,85,'million')
>>> tup2
(50, 69, 85, 'million')
\end{lstlisting}

\textbf{Szótár:} címkék mellé értékeket rendelhetünk, ha lekérdezzük a címkét, visszakapjuk az értéket.
\begin{lstlisting}
>>> dic = {'neve' : 'Pisti', 'apa' : 'Tibi',
'anya' : 'Rozi'}
>>> dic
{'neve': 'Pisti', 'apa': 'Tibi', 'anya': 'Rozi'}
>>> dic['neve']
'Pisti'
>>> del dic['anya']
>>> dic
{'neve': 'Pisti', 'apa': 'Tibi'}
\end{lstlisting}

\textbf{If, else, elif} utasításokkal logikai vizsgálatokat tehetünk: ha a kifejezés értéke teljesül, az \emph{if} utasítás fut le, különben az \emph{else}. Az \emph{elif} utasítás az \emph{else if} rövidítése. Többek között ezért sincs a Pythonban \emph{switch-case} szerkezet. Az alábbi kódrészlet mutatja, hogyan viselkednek ezek az utasítások -- ha az \emph{if} nem teljesül, akkor a futás továbblép a következő \emph{elif}-re. Ha egyik sem teljesül, lefut az \emph{else} utasítás.
\begin{lstlisting}
if kifejezes:
...
elif kifejezes:
...
elif kifejezes:
...
else:
...
\end{lstlisting}

\textbf{Kivételkezelés} segítségével ha fellép egy kivétel a \emph{try} blokkban, valamint le van kezelve egy \emph{except} ágban, akkor a vezérlés ott folytatódik az \emph{except} ágban, nem akad meg a szkript futása. A \emph{try} blokkban hívott függvények által kiváltott kivételt is lekezeli!
\begin{lstlisting}
>>>def kill_all_but_one_space_and_alphabetic(string):
	string = re.sub('^[\t\n\r\f\v\"]+','',string)
	string = re.sub('[\t\n\r\f\v\"]+',' ',string)
	string = re.sub("^[0-9\.]+",'',string)
	return string
>>>try: kill_all_but_one_space_and_alphabetic(string)
>>>except TypeError as reszlet:
	print('Futasi hiba kezelese:',reszlet)
\end{lstlisting}

\textbf{Függvények} is írhatóak természetesen Python segítségével. A \emph{def} kulcsszóval kezdődik definiálása, neve és paraméterlistája, majd ,,:'' után tabulátorral blokkosítva:
\begin{lstlisting}
>>> def function(nev):
	print 'hello: %s' % nev
>>> function('Ivett')
hello: Ivett
\end{lstlisting}
Alapértelemezett érték is adható függvény paramétereinek, de csakis balról jobbra haladva. Ezt a függvény meghívásakor is be kell tartani!
\begin{lstlisting}
>>> def osszead(x=0,y=0):
	return x+y
>>> osszead() =>0
>>> osszead(-1,5) => 4
>>> osszead(5) => 5
>>> osszead(,-1)
SyntaxError: invalid syntax
\end{lstlisting}

\subsubsection{Objektum orientált}
A Pythonban minden adattípus objektum, a beépített típusokat is bővítheti a felhasználó! \textbf{Minden operátor felüldefiniálható} speciális nevű tagfüggvényekben.
Példa az összeadásra és szorzásra:
\begin{lstlisting}
		+	__add__
		*	__mul__
\end{lstlisting}

\textbf{Osztályok} -- egyben be szeretném mutatni az öröklődést, valamint azt, hogy a gyermekosztály felüldefiniálhatja, kibővítheti örökölt attribútumait/osztályait. Minden tagfüggvénynek \emph{self} paraméterrel kell kezdődnie, ez referencia saját magára, ezzel lehet a tagfüggvényen belül hivatkozni az adott objektum elemeire.
\begin{lstlisting}
>>> class parentClass:
	var1 = "string"
	var2 = 5
	def printVars(self):
		print 'var1: ' + self.var1 + '\t' +
		+ 'var2:' + str(self.var2)
>>> obj1 = parentClass()
>>> obj1.printVars() var1: string var2: 5
\end{lstlisting}

\textbf{Gyermekosztály}: minden objektum rendelkezik egy \_\_init\_\_(self) beépített konstruktorral is, amit itt felüldefiniálok. Ez lefut aamutomatikusan mindig, amikor létrehozom ennek az osztálynak egy példányát. Ezen kívül van még \_\_del\_\_ tagfüggvény is, ami az objektum megsemmisülésekor fut le.
\begin{lstlisting}
>>> class childClass(parentClass):
	var2 = "more string"
	def __init__(self):
		print "Jo napot kivanok mindenkinek!"
>>> obj2= childClass()
	Jo napot kivanok mindenkinek!
>>> obj2.printVars()
	var1: string var2: more string
\end{lstlisting}
%
\newpage
\section{Elvégzett munka ismertetése}
A gyakorlat első felében egy fejletlenebb Word átalakítót (Antiword) használtam, azonban annak teszteléskor kiderült, hogy hiányosan konvertálja az állományokat. Előnye volt, hogy platformfüggetlen megoldás. Ellenben még akkor készült, amikor a Word még zárt formátum volt, ezért nem lehetett hozzá olyan szabad feldolgozót írni, ami kihasználhatta volna a Microsoft megoldásait. Azonban 2008-ban nyílttá tették, azóta a LibreOffice/OpenOffice megfelelően formázottan jeleníti meg azokat. Így itt csak ennek a megoldásait ismertetem.\\
\textbf{Használt konfiguráció:}
	\begin{itemize}
		\item Ubuntu 10.10 \href{http://releases.ubuntu.com/10.10/}{http://releases.ubuntu.com/10.10/}
		\item Python 2.7.3 \href{http://www.python.org/download/}{http://www.python.org/download/}
		\item Beautiful Soup 4.1.0 \href{http://www.crummy.com/software/BeautifulSoup/}{http://www.crummy.com/software/BeautifulSoup/}
		\item LibreOffice 3.3.4 \href{http://www.libreoffice.org/}{http://www.libreoffice.org/}
		\item PyODConverter 1.2  -- \href{https://github.com/mirkonasato/pyodconverter}{https://github.com/mirkonasato/pyodconverter}
	\end{itemize}
				
	A fejlesztés folyamán megkötéseket kellett tenni az operációs rendszer függetlenségét tekintve: a fejlesztést egy fejlettebb Python verzióban végeztem, míg a LibreOffice/OpenOffice egy régebbi változatot használ, illetve nincs külön telepítő a Python-Uno hídhoz Windowson. Ezért Windowson nehéz beállítani a PyUno Bridge megfelelő használatához szükséges környezetet, így a doc-html konverziót \textbf{Ubuntu 10.10}-ben végeztem. Ezt is ajánlom a futtatáshoz (Ubuntu 10.10+). A futtatáskor ugyanazok a lépések egyébként mindkét operációs rendszer tekintetében konverziótól eltekintve. Tehát ha kivesszük a szkriptből a html átalakításért felelős részt, ugyanúgy működik bármilyen operációs rendszerben, ami képes Python szkript futtatására, és fel van telepítve megfelelően a LibreOffice/OpenOffice.
	
	Először a Python-t kell telepíteni (2.7.1+), majd Windows esetében a Python fő könyvtárát a környezeti változókhoz (PATH). Következő lépés a Beautiful Soup modul hozzáadása a Python-hoz, mely legegyszerűbben a forrás segítségével telepíthető. Letöltése után kicsomagolva futtatni kell az ,,setup.py''-t, mely feltelepíti a kiegészítőt:
\begin{verbatim}
	python setup.py install
\end{verbatim}

	Következő lépés feltelepíteni a LibreOffice/OpenOffice-t. Ellenőrizni a következők meglétét: ,,python-uno'' csomag, illetve szkript -- ,,DocumentConverter.py'' (PyODConverter szkript), ,,parser.py'' (a feldolgozást végző szkript).
\subsection{Használat}
	A szkript futtatásához először egy könyvtárban kell lenniük:
	\begin{itemize}
		\item ,,DocumentConverter.py''
		\item ,,parser.py''
		\item ,,.doc'' állományok
	\end{itemize}
	A Terminálban először abba a könyvtárba kell navigálni, ahol az állományok találhatóak. A program futtatása a következő paranccsal történik:
	\begin{verbatim}
		python parser.py
	\end{verbatim}
	A futás eredményeként létrejön a html változat is, a dokumentumhoz tartozó képekkel. Továbbá létrejön a \emph{,,Statistic\_Requirement.csv''}. Ez tartalmazza tabulátorral elválasztva a TR-ek adatait. Megjelenítése feldolgozása lehetséges táblázatkezelő rendszerekkel (Microsoft Excel, LibreOffice Calc), valamint adatbázisba is importálható.
	
	CSV-t Excelben megjeleníteni a Data/Import external data/Import Data menüponton keresztül lehetséges. Elválasztónak alapértelmezetten tabulátor van beállítva, de tetszőleges elválasztó beállítható a szkriptben a \emph{,,glue''} változó átírásával. LibreOffice-nál is hasonló módon lehet importálni, vagy a megnyitáskor beállítani az elválasztót. \\
	Így írom ki a megadott kimenetre egy tömb elemeit, melyet a \emph{,,glue''} karakterlánccal illesztek össze egy karakterlánccá:
	\begin{verbatim}
		# Output
		fout_plain=open("Statistic_Requirement.csv","w")
		# CSV separator
		glue = '\t'
		fout_plain.write(glue.join(SR)+'\n')
		fout_plain.close()
	\end{verbatim}
	Az eredmények természetesen erősen függnek a dokumentum jól formázottságától is, olyan TR-ek nem jelennek meg a táblázatban, amikhez nem töltötték ki jól a TR számhoz tartozó mezőt. \emph{,,TR number with link to MHWeb (TR state / Answer code / explanation)''} cella tartalmát vizsgálja, ha ott talál érvényes TR-t, kiírja fájlba.
	
	A tesztesetek és TR-ek érzékeny információk, ezért azok helyett nem valós értékeket helyeztem el.
	\begin{figure}[bth]
		\centering
		\resizebox{5cm}{!}{\includegraphics{hw_title.png}}
		\caption{HW környezet és a teszt típusa}
	\end{figure}
	\begin{figure}[bth]
		\centering
		\resizebox{13.5cm}{!}{\includegraphics{TRs.png}}
		\caption{TR-okat tartalmazó táblázat}
	\end{figure}
\begin{landscape}
\hfill
\begin{figure}[tbh]
		\centering
		\resizebox{20cm}{!}{\includegraphics{csv.png}}
		\caption{CSV kimenet részlet}
\end{figure}
\end{landscape}
\subsection{TR adatai -- adatstruktúra}
\textbf{Az alábbi adatok jelennek meg minden TR-hez:}
\begin{itemize}
	\item TC ID
	\item TC szlogen
	\item Rövid leírása a problémának
	\item TR szám
	\item TR hibás termék MHO
	\item Hatása a forgalomra (Igen/Nem) [típus, \%]
	\item (új/régi) / hibák száma
	\item Miért kellett kiküldeni / komment / magyarázat / TR komolysága (SEB-I,C)
	\item NSP típus
	\item Teszt típus
	\begin{itemize}
		\item Rollback
		\item Network Redundancy
		\item NM Subsystem Regression
		\item Maiden Installation/Wizard
		\item Dummy Upgrade
		\item Upgrade
		\item LRT on target
		\item SIM LRT (MAIA/MAIA-Light/32 bit/64-bit/Dicos-Linux/Dicos-Dicos)
	\end{itemize}
\end{itemize}
\subsection{Alkalmazáslogika}
	A szkript váza alapvetően a doc/html átalakító meghívásából, valamint három egymásba ágyazott \emph{for} ciklusból áll. Az első ciklusban végigmegy az összes mappában található html állományon, a másodikban kikeresi a beolvasott html-ből a TR-eket tartalmazó táblázatokat, és a táblázatokban nem található adatokat is kikeresi a dokumentum adott fejezetéből. A harmadikban végigmegy az egyes sorokon. Itt kikeresi az összes táblát, amiben található TR:
	\begin{lstlisting}
	errors = soup.find_all(text=re.compile(
	"^.*?TR(\s)*?number(\s)*?with(\s)*?link.*?"))
	\end{lstlisting}
	Látható minden lekérdezésen, hogy a reguláris kifejezések (regular expression) használata nem megkerülhető. Következő példa pedig a HW, azaz futtatási környezeteket keresi ki, mert több is megadható egy teszttípushoz, ahol vannak a TR-ek:
	\begin{lstlisting}
	hws = soup.find_all(text=re.compile(
	"^.*?Used(\s)*?HW(\s)*?configuration.*?"))
	\end{lstlisting}
	\emph{stripped\_strings} generátor (csak azokat az elemeket adja vissza, amik megfelelnek a feltételeknek -- itt mindent, csak whitespace karaktereket nem) elvágja a cella tartalmát a whitespace karaktereknél: egy tömböt ad vissza, ezzel kikerülöm a Word által beszúrt sok fölösleges sortörést, tabulátort, formázást. Ennek elemeit adom meg egy tömbnek, ami a megformázva egy sorát adja majd a CSV állománynak, ami megfelel egy TR-nek és környezetének.
	\begin{lstlisting}
for error in errors:
	...
	for tr in 
	error.find_parent("table").find_all("tr"):
		...
		tds = tr.find_all("td")
		for index,td in enumerate(tds):
			for st in td.stripped_strings:
			adatok kinyerese tombbe
	\end{lstlisting}
	A \emph{kill\_all\_but\_one\_space\_and\_alphabetic(string)} függvény jól mutatja, hogy a mintaillesztés egyik nagy akadálya volt a Word által elhelyezett számtalan felesleges sortörés, tabulátor, azaz whitespace karakter. Először a karakterlánc elejéről leszedi az összes ilyen elemet, a közepéből pedig minden egyes előfordulás esetén (ha több is van egymás után) kicseréli egyetlen szóközre. Valamint a címek esetére hasznos volt a számozást is eltávolítani.
	\begin{lstlisting}
def kill_all_but_one_space_and_alphabetic(string):
string = re.sub('^[\t\n\r\f\v\"]+','',string)
string = re.sub('[\t\n\r\f\v\"]+',' ',string)
string = re.sub("^[0-9\.]+",'',string)
return string
	\end{lstlisting}
A \emph{get\_test\_type(title)} függvény mintaillesztéssel kikeresi a TR-hez tartozó teszttípust, majd azt adja vissza. Részlete:
	%
	\begin{lstlisting}
def get_test_type(title):
	title = (re.split(" on",title))[0]
	if re.match('^.*rollback.*',title,re.I):
		return "Rollback"
	elif re.match('^.*?network(\s)*?
redundancy.*?',title,re.I):
		return 'Network Redundancy'
	elif 
re.match('^.*?nm(\s)*?subsystem(
\s)*?regression.*?',title,re.I):
		return 'NM Subsystem Regression'
	...
	\end{lstlisting}
	
	\textbf{Rowspan}: egy cella tartalma tartozhat több sorhoz is, ezért ezt a cella tartalma, pozíciója, ismétlődési száma eltárolódik a \emph{rowspan\_iter}, \emph{rowspan}, \emph{rowspan\_poz}, változókba -- több is lehet egy táblázatban, ezért tömbben kerülnek elhelyezésre. A first\_time tömb azt tartalmazza, hogy az i. elem először kerül e kiírásra. Erre azért volt szükség, mert külöben nem kerülne be az első sorba ahol előfordul, mivel másik <tr> tagba kerül a sor többi cellájától. Az alábbi kódrészletben helyezem el a kiírni kívánt rekord adatai között.
%
\begin{lstlisting}
for i in range(0,len(rowspan)):
	if rowspan_iter[i] >= 1 and 
first_time[i] == False:
		rowspan_iter[i] -= 1
		TC[rowspan_poz[i]:rowspan_poz[i]] 
		= [rowspan[i]]
	elif first_time[i] == True:
		first_time[i] = False
\end{lstlisting}

	Egy cellában több TR is található, ezeket szétválasztja a szkript, majd a sor összes adatát hozzárendeli az összes TR-hez, majd kiírja fájlba azokat. Valamint reguláris kifejezés alapján szűri meg az érvényes TR számmal rendelkező rekordokat a nem relevánsaktól -- két betűt 5 szám követ vagy SDP-vel kezdődik a sztring. A \emph{non\_utf} minden nem ascii karaktert kiszűr, mivel azoknak karakterkódja nagyobb 127-nél, így azokra pedig nem fut le a beépített \emph{for} ciklus (nem teljesül a futási feltétel).
	\begin{lstlisting}
if re.match('((^[a-z][a-z][0-9][0-9]
[0-9][0-9][0-9])|(^SDP)).*?',TC[3],re.I):
	more_TR = re.split(" ", TC[3])
	for i in range(0,len(more_TR)):
		if len(more_TR[i])== 7:
			temp = []
			temp.append(TC[0])
			temp.append(TC[1])
			temp.append(TC[2])
			temp.append(more_TR[i])
			temp.append(TC[4])
			temp.append(TC[5])
			temp.append(TC[6])
			temp.append(TC[7])
			temp.append(TC[8])
			temp.append(TC[9])
			row=glue.join(temp)
non_utf = ''.join([x for x in row if ord(x) < 128])
fout_plain.write(non_utf + '\n')
	\end{lstlisting}
%
\section{Továbbfejlesztési tervek}
\textbf{A megbeszélésen a további fejlesztési javaslatokat tették:}\\
Ki kell bővíteni egy komplexebb rendszer részeként. Melynek végeredményeként: hibakezelés fejlesztése -- ehhez több idő és dokumentum kell tesztelésre. Hyperlinkekkel  kell ellátni a TR-eket, állapotukat online frissíteni.
%
\section{Tapasztalatok, vélemények}
Egyszerű hallgatóként is befogadtak, kedvesen álltak hozzám a teljes részlegen ahová kerültem. Munkám végeztével gyakornoki állást ajánlottak, amit örömmel elfogadtam. Itt készítem diplomamunkámat is. Inspiráló munkakörnyezet, jó munkamorál jellemzi a céget versenyképes fizetéssel.
%
\newpage
\section{Irodalom, és csatlakozó dokumentumok jegyzéke}
\label{sec:irod}
%\begin{itemize}
%	\item Könyv: szerző, cím dőlt betűkkel, kiadó neve, kiadó városa, kiadás éve
%	\item Újságcikk: szerző, „cím idézőjelek között”, újság neve dőlt betűkkel, évfolyam, szám, oldalak, kiadás éve
%	\item Diplomadolgozat/disszertáció: szerző, cím dőlt betűkkel, oktatási intézmény, város, végzés éve
%	\item Elektronikus forrás: teljes URL (átviteli protokoll megjelölésével), megtekintés időpontja percre pontosan
%\end{itemize}
\begin{thebibliography}{9}
\bibitem{vid} Python Programming Tutorials -- Python oktató videók \href{http://www.youtube.com/course?list=ECEA1FEF17E1E5C0DA}{http://www.youtube.com/course?list=ECEA1FEF17E1E5C0DA}, 2012.08.31. 10:40
\bibitem{soup} Beautiful Soup Documentation \href{http://www.crummy.com/software/BeautifulSoup/bs4/doc/}{http://www.crummy.com/software/BeautifulSoup/bs4/doc/}, 2012.08.31. 10:40
\bibitem{hard} Learn Python The Hard Way, 2. kiadás \href{http://learnpythonthehardway.org/book/}{http://learnpythonthehardway.org/book/}, 2012.08.31. 10:40
\end{thebibliography}
%
\newpage
\section{Munkanapló}
Az alábbi táblázatban található a szakmai gyakorlatom során végzett napi tevékenységem tömör összefoglalása:
%
\begin{table}[h!]
\centering
    \begin{tabular*}{\textwidth}{l||l|l@{\extracolsep{\fill}}}
        \multirow{5}{*}{1. hét} & 2012.08.06. & Ismerkedés a vállalattal, körút az Ericssonnál \\ & 2012.08.07. & Python alapvető szintaxis, laborlátogatás\\ & 2012.08.08. & Ismerkedés a Python nyelvvel -- objektumok, osztályok \\ & 2012.08.09. & Jelölőnyelvek feldolgozására alkalmas modulok vizsgálata \\ & 2012.08.10. & Modulok tesztelése, példa szkriptek előállítása, vizsgálata\\ \hline
		\multirow{6}{*}{2. hét} & \multirow{2}{*}{2012.08.13.} & Megfelelő modul megtalálása a Word dokumentációk \\ & & feldolgozásához, mivel lehet azt jelölőnyelvvé alakítani \\ & 2012.08.14. & Modulok, megoldások tesztelése \\ & 2012.08.15. & Antiword kimetének feldolgozása \\ & 2012.08.16. & Antiword feldolgozott kimenetének vizsgálata \\ & 2012.08.17. & Eredmények összevetése az eredeti táblázattal \\ \hline
		\multirow{5}{*}{3. hét} & 2012.08.20. & Hibás konvertáló modul elvetése, új megoldás keresése \\ & 2012.08.21. & Beautiful Soup modul alkalmazása \\ & 2012.08.22. & Alapvető navigálás a html fában a modullal \\ & 2012.08.23. & Trouble Riportok (TR) kinyerése \\ & 2012.08.24. & TR modulokhoz tartozó környezet kinyerése \\ \hline
		\multirow{5}{*}{4. hét} & 2012.08.27. & TR modulokhoz tartozó további adatok kinyerése \\ & 2012.08.28. & Szkript tesztelése, eredmények összevetése az eredetivel \\ & 2012.08.29. & Rowspan probléma kiküszöbölése \\ & 2012.08.30. & Több TR egy mezőben probléma megoldása \\ & 2012.08.31. & Prezentáció megtartása, program átadása
    \end{tabular*}
\end{table}
\\[40pt] Vállalati konzulens: \hfill .......................................................
%
\newpage
\section{Időbeosztás részletezése}
Az alábbi táblazatban számolok el a 160 kontaktórával:
%
\begin{table}[h!]
\centering
    \begin{tabular}{l|r}
        \emph{Ismerkedés a Python nyelvvel és a Beautiful Soup modullal} & 40 óra \\ \hline
		\emph{Dokumentációk átalakítása, átalakítók tesztelése}& 18 óra \\ \hline
		\emph{Szkript elkésztése, tesztelése} & 80 óra \\ \hline
		\emph{Vállalati infrastruktúra megismerése, laborlátogatás} & 6 óra  \\ \hline
		\emph{Szkript átadásra előkészítése, prezentációra való felkészülés} & 8 óra \\ \hline
		\emph{Elvégzett munka ismertetése, szkript bemutatása,} & \multirow{2}{*}{8 óra} \\ 
					\emph{Python képzés megtartása} \\ \hline \hline
		\textbf{Szum} & 160 óra \\ \hline
    \end{tabular}
\end{table}
\\[40pt] Vállalati konzulens: \hfill .......................................................
\end{document}